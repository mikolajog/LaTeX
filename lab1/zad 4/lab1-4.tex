\documentclass[a4paper,11pt]{article}
\usepackage[polish]{babel}
\usepackage[utf8]{inputenc}   % lub utf8
\usepackage[T1]{fontenc}
\usepackage{graphicx}
\usepackage{anysize}
\usepackage{enumerate}
\usepackage{times}
 
%\marginsize{left}{right}{top}{bottom}
\marginsize{3cm}{3cm}{3cm}{3cm}
\sloppy


 
\begin{document}

\subsection{\textbf{The Model-Checking Process}} 

In applying model checking to a design the following different phases can be distinguished:
\begin{itemize} 
\item \textit{Modeling} phase:
\begin{itemize} 
\item[--] model 
the system under consideration using the model description language of the model checker at hand; 
\item[--] as a first sanity check and quick assessment of the model perform some simulations; 
\item[--] formalize the property to be checked using the property specification language. 
\end{itemize}
\item \textit{Running} phase: run the model checker to check the validity of the property in the system model. 
\item \textit{Analysis} phase: 
\begin{itemize}
\item[--]property satisfied? $\rightarrow$ check next property (if any); 
\item[--]property violated? $\rightarrow$
\begin{enumerate} 
\item analyze generated counterexample by simulation; 
\item refine the model, design, or property; 
\item repeat the entire procedure. 
\end{enumerate}
\item[--]out of memory? $\rightarrow$ try to reduce the model and try again.
\end{itemize}
\end{itemize}
\end{document}