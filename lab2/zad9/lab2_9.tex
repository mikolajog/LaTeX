\documentclass[a4paper,12pt]{article}      
\usepackage[utf8]{inputenc} 
\usepackage[T1]{fontenc}
\usepackage{times}
\usepackage{amssymb}
\usepackage{amsthm}
\usepackage[polish]{babel}
\theoremstyle{definition}
\newtheorem{ex}{Przykład}
\setcounter{ex}{7}
\sloppy

\begin{document}
\begin{ex}
Wykażemy, że funkcje
\end{ex}
\begin{center}
$ f(x) = -arctgx$ \quad i \quad $g(x) = arccos\frac{x}{\sqrt{1 + x^{2}}}$
\end{center}
różnią się jedynie o stałą $B = -\frac{\pi}{2}$\\

Dla każdego $x \in \mathbb{R}$ mamy: 
$$f'(x) = \frac{-1}{1 + x^2}, $$
$$g'(x) = \frac{-1}{\sqrt{1-{(\frac{x}{\sqrt{1 + x^{2}}})}^{2}}} \cdot \frac{\sqrt{1 + x^{2}} -  \frac{2x^{2}}{2\sqrt{1 + x^{2}}}}{1 + x^{2}} = \frac{-1}{1 + x^{2}}; $$
oznacza to, że: 
$$f'(x) = g'(x),$$
więc na podstawie statniego wniosku możemy napisać:
$$\forall x \in \mathbb{R}:\quad f(x) = g(x) + B$$
Jednocześnie, np. dla $x = 0$ mamy: 
$$ f(0) = 0, \quad g(0)=\frac{\pi}{2},$$
zatem nietrudno zauważyć, że ostatnie równość ma miejsce, gdy $B = -\frac{\pi}{2}$. 
\end{document}