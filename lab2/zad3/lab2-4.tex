\documentclass[a4paper,12pt]{article}
\usepackage[T1]{fontenc}
\usepackage[latin2]{inputenc}
\usepackage[polish]{babel}
\usepackage{amssymb}
\usepackage{amsthm}
\usepackage{times}
\usepackage{anysize}

\marginsize{1.5cm}{1.5cm}{1.5cm}{1.5cm}
\sloppy 

\begin{document}

Istnieje �cis�y zwi�zek mi�dzy rozk�adem macierzy ${\displaystyle A}$ na macierze ${\displaystyle L}$ i ${\displaystyle U}$ a metod� eliminacji Gaussa. Mo�na wykaza�, �e elementy kolejnych kolumn macierzy ${\displaystyle L}$ s� r�wne wsp�czynnikom przez kt�re mno�one s� w kolejnych krokach wiersze uk�adu r�wna� celem dokonania eliminacji niewiadomych w odpowiednich kolumnach. Natomiast macierz ${\displaystyle U}$ jest r�wna macierzy tr�jk�tnej uzyskanej w eliminacji Gaussa.

$$
[A|b] = \left[
\begin{array}{1111}
2&2&4&4\\
1&2&2&4\\
1&4&1&1\\
\end{array}
\right]
= \left[
\begin{array}{1111}
2&2&4&4\\
0&1&0&2\\
0&3&-1&-1\\
\end{array}
\right]
=\left[
\begin{array}{1111}
2&2&4&4\\
0&1&0&2\\
0&0&-1&-7\\
\end{array}
\right]
$$
$$
L = \left[
\begin{array}{111}
1&0&0\\
\frac{1}{2}&1&0\\
\frac{1}{2}&3&1\\
\end{array}
\right] \hspace{1cm}
U = \left[
\begin{array}{1111}
2&2&4&4\\
0&1&0&2\\
0&0&-1&-7\\
\end{array}
\right]
$$


\end{document}