\documentclass[a4paper,12pt]{article}      
\usepackage[utf8]{inputenc} 
\usepackage[T1]{fontenc}
\usepackage{times}
\usepackage{amssymb}
\usepackage{amsthm}
\usepackage[polish]{babel}
\theoremstyle{definition}
\newtheorem{df}{Przykład}
\setcounter{df}{15}
\sloppy

\begin{document}
\begin{df}
Rozwiążemy układ równań
\end{df}
$$ \left\{
\begin{array}{rrrrrrl}
2x_{1} & - & x_{2} & + & 3x_{3} & = & -3,\\
x_{1} & - & x_{2} & + & 2x_{3} & = & -3,\\
x_{1} & - & x_{2} & - & 3x_{3} & = & 0.\\
\end{array}
\right.
$$
Obliczymy najpierw wyznacznik główny $W=det \textbf{A}$ tego układu: 
$$ W = det \left( \left[
\begin{array}{111}
2&-1&3\\
1&-1&2\\
1&-1&-1\\
\end{array}
\right]
\right)
=
\left|
\begin{array}{111}
2&-1&3\\
1&-1&2\\
1&-1&-1\\
\end{array}
\right|
\begin{array}{11}
2&-1\\
1&-1\\
1&-1\\
\end{array}
= 2-2-3+3+4-1=3
$$

\end{document}

